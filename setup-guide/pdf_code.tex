\documentclass[a4paper,12pt]{article}
\usepackage[utf8]{inputenc}
\usepackage[T1]{fontenc}
\usepackage{geometry}
\geometry{margin=1in}
\usepackage{sectsty}
\usepackage{titlesec}
\usepackage{enumitem}
\usepackage{hyperref}
\hypersetup{colorlinks=true, linkcolor=blue, urlcolor=blue}
\usepackage{parskip}
\setlength{\parskip}{1em}
\usepackage{listings}
\lstset{
  basicstyle=\ttfamily\small,
  breaklines=true,
  columns=fullflexible,
  frame=single,
  numbers=none,
  showspaces=false,
  showstringspaces=false
}
\usepackage{tocloft}
\renewcommand{\cftsecleader}{\cftdotfill{\cftdotsep}}
\usepackage[sc]{mathpazo}
\linespread{1.05}

\begin{document}

\begin{titlepage}
  \centering
  \vspace*{2cm}
  \Huge\textbf{Oracle Database Setup with Docker \\} 
  \vspace{1cm}
  \Large A Step-by-Step Guide for Installation and SQL Execution
  \vspace{2cm}
  \normalsize Get Hands-On Experience with Oracle Database and Object-Relational Mapping 
  \vspace{1cm}
  \normalsize Copyright \textcopyright\ 2025 - All right reserved by Pavan Uthsara
  \vfill
\end{titlepage}

\tableofcontents
\newpage

\section{Introduction}
This guide provides a comprehensive specification for installing Oracle Database 23ai Free (the latest free developer edition, superseding Oracle XE) in a Docker container on your laptop. It covers setting up the database, installing GUI tools (Oracle SQL Developer and DBeaver), and executing SQL commands for hands-on experience, including object-relational mapping (ORM). The setup uses a community-maintained Docker image (\texttt{gvenzl/oracle-free:23-full}), which is free and requires no Oracle account.

\subsection{Assumptions}
\begin{itemize}
  \item Operating System: Windows, macOS, or Linux.
  \item Hardware: Minimum 4GB RAM (8GB recommended), 10GB free disk space.
  \item Internet: Required for downloading Docker images and tools.
  \item Time Estimate: 30--60 minutes, plus download times.
\end{itemize}

\section{Prerequisites}
% Setting up prerequisites
Before starting, ensure the following are installed:
\begin{itemize}
  \item \textbf{Docker Desktop}: Download from \url{https://www.docker.com/products/docker-desktop}. Install and verify with:
    \begin{lstlisting}
docker --version
    \end{lstlisting}
    Expected output: \texttt{Docker version 27.2.0} or similar. Start Docker Desktop.
  \item \textbf{Java JDK 11+}: Required for SQL Developer. Download from \url{https://www.oracle.com/java/technologies/downloads} if needed.
\end{itemize}

\section{Installing Oracle Database in Docker}
% Pulling Docker image
\subsection{Pull the Docker Image}
Open a terminal (Command Prompt/PowerShell on Windows, Terminal on macOS/Linux) and run:
\begin{lstlisting}
docker pull gvenzl/oracle-free:23-full
\end{lstlisting}
\begin{itemize}
  \item Downloads the full edition (~5--6 GB) with sample schemas (e.g., HR) for ORM practice.
  \item Alternatives: \texttt{23-slim} (~2 GB, fewer schemas) or \texttt{latest}.
  \item Wait 5--30 minutes based on your internet speed.
\end{itemize}

% Running the container
\subsection{Run the Docker Container}
Start the container with a secure password (8+ characters, including uppercase, lowercase, digit, special character). Replace \texttt{MySecurePassword123} with your choice:
\begin{lstlisting}
docker run --name oracle-db -p 1521:1521 -e ORACLE_PASSWORD=MySecurePassword123 -d -v oracle-data:/opt/oracle/oradata gvenzl/oracle-free:23-full
\end{lstlisting}
\begin{itemize}
  \item \texttt{--name oracle-db}: Names the container.
  \item \texttt{-p 1521:1521}: Maps Oracle’s port to localhost.
  \item \texttt{-e ORACLE_PASSWORD}: Sets SYS/SYSTEM password.
  \item \texttt{-d}: Runs in background.
  \item \texttt{-v oracle-data:/opt/oracle/oradata}: Persists data.
  \item Optional: Use \texttt{-e ORACLE_RANDOM_PASSWORD=true} for a random password (check logs).
\end{itemize}
Initialization takes 3--10 minutes. Monitor with:
\begin{lstlisting}
docker logs -f oracle-db
\end{lstlisting}
Look for \texttt{DATABASE IS READY TO USE!}.

% Verifying container status
\subsection{Verify Container Status}
Check if the container is running:
\begin{lstlisting}
docker ps
\end{lstlisting}
Confirm \texttt{oracle-db} is listed with status \texttt{Up}. If not:
\begin{itemize}
  \item Port conflict: Use \texttt{-p 1522:1521} and adjust connections.
  \item Memory issues: Allocate 4GB+ RAM in Docker settings.
  \item macOS M-series: Use Colima (\texttt{brew install colima; colima start --arch x86_64 --memory 4}).
\end{itemize}

% Creating a sample user
\subsection{Create a Sample Application User}
Create a non-admin user for ORM practice:
\begin{lstlisting}
docker exec -it oracle-db sqlplus sys/MySecurePassword123@localhost:1521/FREEPDB1 as sysdba
\end{lstlisting}
In SQL*Plus:
\begin{lstlisting}
CREATE USER appuser IDENTIFIED BY app_password;
GRANT CONNECT, RESOURCE TO appuser;
GRANT UNLIMITED TABLESPACE TO appuser;
EXIT;
\end{lstlisting}
Use \texttt{appuser}/\texttt{app_password} for connections.

\section{Setting Up GUI Tools}
% Installing SQL Developer
\subsection{Primary Tool: Oracle SQL Developer}
\subsubsection{Download and Install}
\begin{itemize}
  \item Download from \url{https://www.oracle.com/database/sqldeveloper/technologies/download} (e.g., version 24.1).
  \item Choose ZIP with/without JDK based on Java installation.
  \item Unzip and run:
    \begin{itemize}
      \item Windows: \texttt{sqldeveloper.exe}
      \item macOS/Linux: \texttt{sqldeveloper.sh}
    \end{itemize}
  \item If prompted, provide JDK path (Java 11+).
\end{itemize}

\subsubsection{Connect to the Database}
\begin{itemize}
  \item Open SQL Developer, click \texttt{+} to create a connection.
  \item Settings:
    \begin{itemize}
      \item Connection Name: \texttt{OracleDocker}
      \item Username: \texttt{appuser} (or \texttt{SYS})
      \item Password: \texttt{app_password} (or \texttt{MySecurePassword123})
      \item Role: \texttt{SYSDBA} for SYS; blank otherwise
      \item Connection Type: \texttt{Basic}
      \item Hostname: \texttt{localhost}
      \item Port: \texttt{1521}
      \item Service Name: \texttt{FREEPDB1}
    \end{itemize}
  \item Click \texttt{Test Connection}, then \texttt{Connect} if successful.
\end{itemize}

\subsubsection{Execute SQL Commands}
\begin{itemize}
  \item Open a SQL Worksheet (File > New > SQL Worksheet).
  \item Example commands:
    \begin{lstlisting}
-- Create table for ORM
CREATE TABLE employees (
  id NUMBER PRIMARY KEY,
  name VARCHAR2(100),
  salary NUMBER
);
-- Insert data
INSERT INTO employees VALUES (1, 'John Doe', 50000);
COMMIT;
-- Query
SELECT * FROM employees;
    \end{lstlisting}
  \item Run: \texttt{F9} (single statement) or \texttt{F5} (script).
  \item Browse schemas in the Connections panel.
\end{itemize}

% Installing DBeaver
\subsection{Alternative Tool: DBeaver}
\subsubsection{Download and Install}
\begin{itemize}
  \item Download Community Edition from \url{https://dbeaver.io/download} (e.g., version 24.2).
  \item Install:
    \begin{itemize}
      \item Windows/macOS: Run installer.
      \item Linux: Use DEB/RPM or snap/flatpak.
    \end{itemize}
  \item Add Oracle JDBC driver:
    \begin{itemize}
      \item Download \texttt{ojdbc11.jar} from \url{https://www.oracle.com/database/technologies/appdev/jdbc-downloads.html}.
      \item In DBeaver: Database > Driver Manager > Oracle > Libraries > Add File > Select \texttt{ojdbc11.jar}.
    \end{itemize}
\end{itemize}

\subsubsection{Connect to the Database}
\begin{itemize}
  \item Create new connection: Database > New Database Connection > Oracle.
  \item Settings:
    \begin{itemize}
      \item Host: \texttt{localhost}
      \item Port: \texttt{1521}
      \item Database/SID: Service Name, \texttt{FREEPDB1}
      \item Username: \texttt{appuser} (or \texttt{SYS})
      \item Password: \texttt{app_password} (or \texttt{MySecurePassword123})
    \end{itemize}
  \item Test and connect.
\end{itemize}

\subsubsection{Execute SQL Commands}
\begin{itemize}
  \item Open SQL Editor (right-click connection > SQL Editor).
  \item Run same example commands as above.
  \item Execute: \texttt{Ctrl+Enter} (statement) or \texttt{Ctrl+Shift+Enter} (script).
  \item Use Database Navigator for schema exploration.
\end{itemize}

\section{Executing SQL Commands: Tips}
% General SQL execution tips
\begin{itemize}
  \item \textbf{Basic Commands}:
    \begin{itemize}
      \item Create: \texttt{CREATE TABLE ...}
      \item Insert: \texttt{INSERT INTO ... VALUES ...; COMMIT;}
      \item Query: \texttt{SELECT * FROM ...;}
      \item Update/Delete: \texttt{UPDATE ... SET ...; DELETE FROM ...; COMMIT;}
    \end{itemize}
  \item \textbf{ORM Integration}: Use JDBC URL \texttt{jdbc:oracle:thin:@localhost:1521/FREEPDB1} for Java (Hibernate) or Python (cx\_Oracle/SQLAlchemy).
  \item \textbf{Troubleshooting}:
    \begin{itemize}
      \item Verify container: \texttt{docker ps}
      \item Check logs: \texttt{docker logs oracle-db}
      \item Firewall: Allow port 1521.
      \item Credential errors: Check password case sensitivity.
    \end{itemize}
\end{itemize}

\section{Managing the Container}
% Container management commands
\begin{itemize}
  \item \textbf{Stop}: To stop the container safely:
    \begin{lstlisting}
docker stop oracle-db
    \end{lstlisting}
  \item \textbf{Start}: To restart the container (e.g., after rebooting your computer):
    \begin{lstlisting}
docker start oracle-db
    \end{lstlisting}
  \item \textbf{Remove} (caution: only if you want to delete the container; data persists if using volume):
    \begin{lstlisting}
docker rm -f oracle-db
    \end{lstlisting}
  \item \textbf{Remove Image} (to free disk space after removing container):
    \begin{lstlisting}
docker rmi gvenzl/oracle-free:23-full
    \end{lstlisting}
\end{itemize}

% Starting container after computer restart
\subsection{Starting the Container After Computer Restart}
Since you turn off your computer daily, follow these steps each day to resume working with the Oracle Database:
\begin{enumerate}
  \item \textbf{Start Docker Desktop}: Open Docker Desktop on your computer (Windows, macOS, or Linux). Ensure it’s running before proceeding.
  \item \textbf{Start the Container}: Open a terminal (Command Prompt/PowerShell on Windows, Terminal on macOS/Linux) and run:
    \begin{lstlisting}
docker start oracle-db
    \end{lstlisting}
    \begin{itemize}
      \item This restarts the existing \texttt{oracle-db} container, reusing the persisted data (thanks to the \texttt{-v oracle-data:/opt/oracle/oradata} volume).
      \item Wait 1--3 minutes for the database to become available. Check status:
        \begin{lstlisting}
docker logs oracle-db
        \end{lstlisting}
        Look for \texttt{DATABASE IS READY TO USE!}.
    \end{itemize}
  \item \textbf{Verify}: Confirm the container is running:
    \begin{lstlisting}
docker ps
    \end{lstlisting}
    Ensure \texttt{oracle-db} is listed with status \texttt{Up}.
  \item \textbf{Connect}: Use SQL Developer or DBeaver with the same connection settings (localhost:1521, service name \texttt{FREEPDB1}, user \texttt{appuser}/\texttt{app_password} or \texttt{SYS}/\texttt{MySecurePassword123}).
\end{enumerate}

% Best practices for stopping the container
\subsection{Best Practices for Stopping the Container}
To ensure data integrity and system efficiency when stopping the container:
\begin{itemize}
  \item \textbf{Stop Gracefully}: Always use \texttt{docker stop oracle-db} to allow the database to shut down cleanly, preserving data and avoiding corruption.
  \item \textbf{Avoid Forceful Termination}: Do not use \texttt{docker kill oracle-db} or \texttt{docker rm -f oracle-db} unless absolutely necessary (e.g., container is unresponsive), as this may interrupt database processes.
  \item \textbf{Stop Before Shutdown}: If you plan to shut down your computer, stop the container first:
    \begin{lstlisting}
docker stop oracle-db
    \end{lstlisting}
    This ensures the database closes properly.
  \item \textbf{Monitor Resources}: If you’re not using the database, stop the container to free up memory and CPU (the Oracle container uses ~2GB RAM when running).
  \item \textbf{Backup Data}: The \texttt{oracle-data} volume persists data across restarts. To back up, export the database using SQL Developer (Tools > Database Export) or DBeaver (right-click database > Tools > Backup).
\end{itemize}

\section{Next Steps}
% Suggestions for further learning
\begin{itemize}
  \item Explore sample schemas (e.g., \texttt{SELECT * FROM HR.EMPLOYEES;}).
  \item Build ORM apps with Hibernate (Java) or SQLAlchemy (Python).
  \item Contact for help with error messages.
\end{itemize}

\end{document}